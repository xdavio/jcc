\documentclass{beamer}
\usepackage[latin1]{inputenc}
\usepackage{graphicx}

\mode<presentation>
{
  \usetheme{Warsaw}
  % or ...

  \setbeamercovered{transparent}
  % or whatever (possibly just delete it)
}

\usetheme{default}
\title[STAT 371]{Introduction to Statistics for the Life Sciences}
\date{Spring 2016}
\begin{document}

\section{Introduction}

\begin{frame}
\titlepage
\end{frame}

\begin{frame}
\frametitle{Topics} 
\tableofcontents
\end{frame}

\begin{frame}{Another Definition}
\begin{quote}
``Statistics is the science of understanding data and of making decisions in the face of variability and uncertainty."
\end{quote}
--Statistics for the Life Sciences, page 1.
\end{frame}


\begin{frame}{Example 1.1.1}
24 sheep were vaccinated and 24 were not. All were exposed to Anthrax. 
\input{ex1.tex}
\pause
The vaccine apparently works. What if there were only 2 sheep in each group?
\end{frame}

\begin{frame}{Example 1.1.2}
The relationship between a controlled bacterial infection and liver tumor incidence was studied in a certain species of mice. 
\input{ex2.tex}
\end{frame}

\section{Variability}
\begin{frame}{Variability}
What is variability?
\pause
\begin{itemize}
  \item Heart Rate (bbm)
  \pause
  \item Heart Rate in response to caffeine
\end{itemize}
Loosely speaking, similar things still exhibit differences.
\end{frame}

\begin{frame}{Wikipedia's Definition}
\begin{block}{Definition}Variability is how spread out or closely clustered a set of data is.\end{block}

Variability may refer to:
\begin{itemize}
  \item  Climate variability
  \item  Genetic variability
  \item  Heart rate variability
  \item  Human variability
  \item  Spatial variability
  \item  \textbf{Statistical variability, a measure of dispersion in statistics}
  \item  Variability hypothesis, nineteenth century hypothesis that males have a greater range of ability than females (???)
\end{itemize}
\end{frame}

\subsubsection{Signal Versus Noise}

\begin{frame}{Example 1.1.2}
Recall the tumor/bacteria/mice example:
\input{ex2.tex}
How can we separate presumably random tumor incidence from the effect of the introduced bacteria?
\end{frame}

\begin{frame}
\begin{itemize}
  \item Want: know if presence of bacteria increases tumor incidence
  \item 27 total tumors
  \item Group sizes are 13 and 49 (E.coli and not, respectively)
  \item \textbf{Assume bacteria doesn't affect tumor incidence}
  \item Then, how plausible are the tumor incidence proportions .62, .39?
\end{itemize}
\end{frame}


\begin{frame}{Assess Plausibility}
\begin{itemize}
  \item Randomly assign the 27 tumors to two groups of size 13, 49.
  \item Repeat this $10^5$ times.
  \item How does the tumor incidence of the bacteria group look?
\pause
  \item 3.8\% of the time, the prop. of tumors in the bacteria group is .62 or more.
  \item Most fundamental point in this course
  \item The signal seems to win out over the noise.
\pause
  \item What if the 3.8\% were 15.2\%?
\end{itemize}
\end{frame}

\begin{frame}[plain]
  \includegraphics[width=.8\textwidth]{ex3.pdf}
\end{frame}


%\section{Design of Experiments}

\section{Evidence}

\subsection{Anecdote}

\begin{frame}{Example 1.2.1}
\begin{itemize}
  \item In 1911 a 65-year-old woman who had been deaf since birth was struck by lightning.  She survived, and her deafness was cured!
  \item An NYT headline surfaced: ``Lightning Cures Deafness."
  \item This is called \textbf{anecdotal evidence}.
\end{itemize}
\pause
\begin{block}{Anecdotal Evidence:} An interesting event that may help scientists generate hypotheses but in and of itself does not establish scientific theory.
\end{block}

\end{frame}

\subsection{Observational Study}

\begin{frame}{Definition}

\begin{itemize}
  \begin{block}{Observational Study:} A study in which the researcher collects data from subjects but is not manipulating the conditions that lead to the measurement of interest.\end{block}
  \item Example 1.2.2: 30 homosexual men, heterosexual men, and heterosexual women had the area of their AC measured.
  \input{ex4.tex}
\end{itemize}

\end{frame}

\begin{frame}{Confounding Variables}

\begin{itemize}
  \item However, 24 of the 30 homosexual men had AIDS (as did 6 of the heterosexual men, and none of the heterosexual women).
  \item This is a problem. Why?
\pause
  \item What if AIDS has a connection with the size of the AC?
\end{itemize}

\begin{block}{Confounding Variable:} A variable that varies with both the independent and dependent variable, making the relationship between the two apparent but not actually valid.\end{block}

\end{frame}

\begin{frame}{Weird Confounds 1}
\begin{itemize}
  \item A study supported the claim that people who married later in life were less likely to develop cognitive impairment. 
  \item Confounding Variable?
\end{itemize}
\end{frame}

% \begin{frame}{Weird Confounds 2: Example 1.2.8}
% \begin{itemize}
%   \item A study was conducted giving medication that attempted to lower cholesterol.
%   \item Many of the subjects took less than the prescribed amount of medication
% \end{itemize}

% 5-year mortality rates:
% \begin{center}
% \begin{tabular}{lll}
% Adherence & Treatment & Placebo \\
% $\geq 80$\% & 15\% & 15.1\% \\
% $< 80$\% & 24.6\% & 28.2\% \\
% \end{tabular}
% \end{center}

% \pause
% The subjects were divided into adherence groups after the experiment.

% \end{frame}



% \subsection{Experiment}

% \begin{frame}{Definition}

% \begin{block}{Experiment:} A study in which the researcher imposes the conditions that lead to the measurement of interest, controlling for sources of variability.\end{block}

% \begin{itemize}
%   \item Better suited for relating independent variables with dependent variables
%   \item Better suited for establishing causality
%   \item Allows elimination of confounding variables
% \end{itemize}
% \end{frame}

% \section{Issues with Experiments}

% \subsection{Placebo}

% \begin{frame}
% \begin{itemize}
%   \item Sugar pills, for $\sim 30$\% of people, are effective as painkillers. 
%   \item The patient must believe it is the treatment.
%   \item Placebo / Nocebo: I shall please / I shall harm. 
% \end{itemize}
% \end{frame}

% \begin{frame}{Famous Case}
% \begin{itemize}
%   \item 1950s: angina pectoris treated by surgical ligation of mammary artery to increase blood flow to heart
%   \item Medical community and patients heartily approved
%   \item Later, animal studies cast doubts
%   \item Human experiment with sham operation $\Rightarrow$ No difference in improvement
%   \item Mammary Artery Ligation no longer used
%   \pause
%   \item Ethical issues?
% \end{itemize}
% \end{frame}


% \subsection{Blinding}

% \begin{frame}
%   \begin{block}{Blinding:} A blind experiment is one in which the experimental subjects are not privy to the their treatment assignment. \end{block}
% \begin{itemize}
%   \item Minimize patient expectations from playing a role in the results
%   \item Thought experiment: Imagine a physician believes taking a certain supplement accelerates bone fracture repair.
%   \item $\Rightarrow$ Can we trust this physician's interpretation of the X-ray?
% \pause
%   \item How to remedy this problem? Double-blind experiments.
% \end{itemize}
% \end{frame}

% \subsection{Double-Blinding}

% \begin{frame}{Definition}
% \begin{block}{Double-Blind Experiment:} An experiment in which both the persons making evaluations and the subjects are not privy to the treatment assignment. \end{block}

% \vspace{20mm}

% Recall the Mammary Ligation example -- it was double-blinded.
% \end{frame}

%historical controls?


%\section{SRS}

\section{Population Versus Sample}

\begin{frame}{Population}
\begin{block}{Population:} The population consists of all the subjects/specimens of interest.\end{block}

Examples:
\begin{itemize}
  \item The fish in a pond
  \item People with autism under the age of 5
  \item Tap water in US cities
\end{itemize}

\pause 

However, we can rarely get our hands on the entire population.
\end{frame}

\begin{frame}{Sample}
\begin{block}{Sample:} The sample is a subset of the population with sample size $n$.\end{block}


Examples:
\begin{itemize}
  \item $n=5$ fish in the pond
  \item $n=12$ people with autism under the age of 5
  \item $n=3000$ vials of tap water from US cities
\end{itemize}

\pause

By nature, a sample paints an incomplete picture of the population. 
\end{frame}

% \begin{frame}
% Consider a simple example. You catch $n=100$ fish of a particular species from the ocean and weigh them. The average weight is 1.2 kg. 

% \vspace{10mm}

% If we weighed all such fish, would the resulting average be 1.2 kg?

% \vspace{10mm}

% \begin{block}{Sampling Error:} The discrepancy between the sample and population is called the sampling error. \end{block}

% \end{frame}

% \begin{frame}{Definition of Simple Random Sampling}

% \begin{block}{Simple Random Sample (SRS):}  A sample of size $n$ from a population such that (a) all subjects/specimens in the population have the same chance of being in the sample and (b) the members of the sample are chosen independently of each other. \end{block}

% \begin{alertblock}{Example}
% Suppose a population of 100 dogs on which SRS with $n=10$ is to be performed. Write the (unique) name of every dog on a piece of paper, put the pieces in a jar, stir, draw 10. 
% \end{alertblock}

% \end{frame}

% \begin{frame}{SRS: In Real Life}
% \begin{alertblock}{Example}
% Suppose a population of 100 dogs on which SRS with $n=10$ is to be performed. Give each dog a unique numerical ID. Use a computer to randomly generate numbers that correspond to these IDs until 10 have been generated. 
% \end{alertblock}

% Note: Humans can't be trusted as random number generators.

% \end{frame}

% \subsection{Bias}

% \begin{frame}
% Volunteer
% \end{frame}


% \begin{frame}{Definition}
% \begin{block}{Bias}
% A biased sample systematically overestimates or underestimates a characteristic of the population
% \end{block}
% \begin{itemize}
%   \item Paid subjects for drug testing
%   \item Brood size sampling
%   \item Other ideas?
% \end{itemize}
% \end{frame}




\end{document}
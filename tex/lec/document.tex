\documentclass[11 pt]{article}
\usepackage{amssymb,amsmath, amsthm, graphics, subfig, listings, graphicx, framed}
%\usepackage{natbib}
%formatting that is good for editing (it double spaces)
\renewcommand{\baselinestretch}{1.4}
\textwidth 6in \textheight 9in \hoffset -0.30in \topmargin -0.45in
\interfootnotelinepenalty=10000 %keeps footnotes on a single page

%comment out
%----------------------
\newcommand{\commentout}[1]{}
%--------------------------

%%%%%%%%%%%%%%%%%%%%%%%%%%%%%%%%%%%%%%%%%%%%%%%%%%%%%%%%%%%%%%%%%%%%%%%%%%%%%%%%%
%PART I. GENERAL COMMANDS

%%%%%%%%%%%%%%%%%%%%%%%%%%%%%%%%%%%%%%%%%%%%%%%%%%%%%%%%%%%%%%%%%
% I. NUMBERING AND ENVIRONMENTS FOR THEOREMS, PROPOSITIONS, LEMMAS, AND EQUATIONS

%A. Actual Theorems
\newtheorem{ass}{Assumption}
\newtheorem{prop}{Proposition}
\newtheorem{fact}{Fact}
\newtheorem{lem}{Lemma}
\newtheorem{claim}{Claim}
\newtheorem{thm}{Theorem}
\newtheorem{cor}{Corollary}
\newtheorem{con}{Conjecture}
\newtheorem{defn}{Definition} %use \textbf{} to set it off
\newtheorem{rem}{Remark}
\newtheorem{rec}{Recall}
\newtheorem{problem}{Problem}
\newtheorem{example}{Example}
\newtheorem{note}{Note}
\newtheorem{question}{Question}
\newtheorem{ques}{\textcolor{red}{Question}}
\newtheorem{com}{\textcolor{red}{Comment}}
\newtheorem{todo}{\textcolor{red}{To Do}}

%B. Special Referencing

%B1. Built in ones

%1. \eqref (equations, in standard package)
 %%2. in commath:
%a)\thmref
%b)\exref (example)
%c) \defnref
%d) \lemref
%e)\propref
%f)\remref
%g)\assref
%h)\colref - corollary
%also figures, section, appendix

%B2. I try to define
%1. fact KEY need dollar signs for it work $\factref{fact:varW}$
\newcommand{\factref}[1]{ \text{Fact~\ref{#1}} }

%%%%%%%%%%%%%%%%%%%%%%%%%%%%%%%%%%%%%%%%%%%%%%%%%%%%%%%%%%%%%%%%%

%Bold Greek
%%%%%%%%%%%%%%%%%%%%%%%%%%%%%%%%%%%%%%%%%%%%%%%%%%%%%%%%%%%%%%%%%
\def\bbeta{\mbox{\boldmath $\beta$}}
\def\bmu{\mbox{\boldmath $\mu$}}
\def\etab{\mbox{\boldmath $\eta$}}
\def\balpha{\mbox{\boldmath $\alpha$}}
\def\btau{\mbox{\boldmath $\tau$}}
\def\bDelta{\mbox{\boldmath $\Delta$}}
\def\bGamma{\mbox{\boldmath $\Gamma$}}
\def\bgamma{\mbox{\boldmath $\gamma$}}
\def\bOmega{\mbox{\boldmath $\Omega$}}
\def\bPsi{\mbox{\boldmath $U$}}
\def\bpsi{\mbox{\boldmath $\mu$}}
\def\bXi{\mbox{\boldmath $\Xi$}}
\def\bxi{\mbox{\boldmath $\xi$}}
\def\bSigma{\mbox{\boldmath $\Sigma$}}
\def\bLambda{\mbox{\boldmath $\Lambda$}}
\def\btheta{\mbox{\boldmath $\theta$}}
\def\bDelta{\mbox{\boldmath $\Delta$}}
\def\bTheta{\mbox{\boldmath $\Theta$}}
\def\etaz{\mbox{\boldmath $\eta$}}

\def\boldX{\mbox{\boldmath $X$}}
\def\boldx{\mbox{\boldmath $X$}}
%%%%%%%%%%%%%%%%%%%%%%%%%%%%%%%%%%%%%%%%%%%%%%%%%%%%%%%%%%%%%%%%%

%%%%%%%%%%%%%%%%%%%%%%%%%%%%%%%%%%%%%%%%%%%%%%%%%%%%%%%%%%%%%%%%%
% III. SHORTCUTS

%A. Greek symbols
\newcommand{\be}{\begin{eqnarray*}}
\newcommand{\ee}{\end{eqnarray*}}
\newcommand{\ff}{\infty}
\newcommand{\ra}{\rightarrow}
\newcommand{\ep}{\epsilon}
\newcommand{\ga}{\gamma}
\newcommand{\al}{\alpha}
\newcommand{\la}{\lambda}
\newcommand{\si}{\sigma}
\renewcommand{\th}{\theta}
\newcommand{\Epos}{E_{\theta|\boldX}}
\newcommand{\Ej}{E_{\theta,\boldX}}
%B. Other Math Symbols

%1. transpose, you want a consistent use, bc you could use t,T or \prime
\def\tran{\mathop{ t }}

%C. Commands
\newcommand{\xra}[1]{\mathop{ \xrightarrow{#1} }}

%D. Fences puts items in the correct fence sizes (parantheses, brackets, etc..)
% (get rid of? look at commath package?)
%fp = fence parentheses
\newcommand{\fp}[1]{ \mathop{ \left( #1 \right) } }
%fb = fence brackets
\newcommand{\fb}[1]{ \mathop{ \left[ #1 \right] } }
%fbr = fence braces meaning \{
\newcommand{\fbr}[1]{ \mathop{ \left\{ #1 \right\} } }

%%%%%%%%%%%%%%%%%%%%%%%%%%%%%%%%%%%%%%%%%%%%%%%%%%%%%%%%%%%%%%%%%


%%%%%%%%%%%%%%%%%%%%%%%%%%%%%%%%%%%%%%%%%%%%%%%%%%%%%%%%%%%%%%%%%
%V. Other Commands

 %command that is called by \Title{Text} and this centers it.
\newcommand{\Title}[1]{\begin{center}{\Large \bf #1} \end{center}}

%%%%%%%%%%%%%%%%%%%%%%%%%%%%%%%%%%%%%%%%%%%%%%%%%%%%%%%%%%%%%%%%%
%END PART I
\newcommand{\xb}{\bar{x}}
\newcommand{\Xb}{\bar{X}}
\newcommand{\sn}{s / \sqrt{n}}
\newcommand{\Sn}{S / \sqrt{n}}
\newcommand{\ho}{\mathcal{H}_0}
\newcommand{\hono}{\mathcal{H}_0}
\newcommand{\mun}{\mu_0}
\newcommand{\ha}{\mathcal{H}_A}
\newcommand{\zc}{z_{\alpha}}
\newcommand{\zctwo}{z_{\alpha/2}}
\newcommand{\tc}{t_{n-1,\alpha}}
\newcommand{\tctwo}{t_{n-1,\alpha/2}}


\begin{document}
\Title{Summary Recap of Lec 14 3/03}
\Title{John Davis}


\section*{Pvalue Calculatios based on the form of $\ha$}

Here, I provide three forms of the alternative hypothesis, the corresponding rejection region, and the corresponding pvalue calculation. Note that in the following $z$ is the observed value of the test statistic. 

\begin{enumerate}
  \item $\ha: \mu < \mun$: In this case, the critical value is $\zc$, and the rejection region is
\[
z < -\zc.
\]
\begin{framed}
The pvalue calculation is $P(Z < z)$.
\end{framed}

  \item $\ha: \mu \neq \mun$: In this case, the critical value is $\zctwo$, and the rejection region is
\[
z > \zctwo \qquad \text{or} \qquad z < - \zctwo
\]
\begin{framed}
The pvalue calculation is $P(Z < -|z|) + P(Z > |z|)$.
\end{framed}

  \item $\ha: \mu > \mun$: In this case, the critical value is $\zc$, and the rejection region is
\[
z > \zc
\]
\begin{framed}
The pvalue calculation is $P(Z > z)$.
\end{framed}
\end{enumerate}

I recommend drawing pictures of areas under the standard normal curve to match up with the pvalue calculations. The pvalue calculations for the $t$-test are analagous. 

\section*{Power Calculation}

Suppose that $\sigma$ is known, the data are normal, $\ho: \mu = \mun$, $\ha: \mu > \mun$, and we know $\mu = \mu_A$. We can then calculate the power.
\[
power = P \left ( Z > z_{\alpha} + \frac{\mun - \mu_A}{\sigma / \sqrt{n}} \right ) 
\]
Alternatively, if $\ha: \mu < \mun$
\[
power = P \left ( Z < -z_{\alpha} + \frac{\mun - \mu_A}{\sigma / \sqrt{n}} \right ) 
\]
For the purposes of this course, we'll ignore the case where $\ha: \mu \neq \mun$


\section*{Sample Size Calculation for Power}
Let $\ho: \mu = \mun$ and $\ha : \mu \neq \mun$. Suppose we know that $\mu = \mu_A$. Then for normal data with known $\sigma$, we can find the sample size to guarantee power = $1 - \beta$:
\begin{framed}
\[
n = \left ( \frac{\sigma (z_{\beta} + z_{\alpha / 2})}{\mun - \mu_A} \right ) ^2
\]
\end{framed}
Remember to round $n$ up to the nearest whole integer. If the form of $\ha$ is $<$ or $>$, then use $\alpha$ instead of $\alpha / 2$ in the formula.

\section*{Estimating the pvalue of the $t$-test using the $t$-table}

We did an extensive example in class about how to do this kind of calculation. Next week, in discussion, a similar calculation will occur.  


\end{document}
